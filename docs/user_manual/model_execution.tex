\section{Model Execution}
\label{sec:model_execution}

FLAME contains a parser program called `xparser' that parses a model
XMML definition into simulation program source code that can be compiled
together with the agent functions implementation source code. The xparser
includes template files which are used to generate the simulation program source code.

The xparser takes as parameters the location of the model file and an option
for serial or parallel (MPI) version, serial being the default if the option is
not specified. A production mode can also be switched on which removes any
debugging features in the simulation program.

\begin{verbatim}
  xparser (Version 0.16.2)
  Usage: xparser [XMML file] [-s | -p] [-f]
          -s      Serial mode
          -p      Parallel mode
          -f      Final production mode
\end{verbatim}

\subsection{Generated Files}

The xparser then generates simulation source code files in the same directory
as the model file. These files include documentation of the model:

\begin{itemize}
  \item stategraph.dot -- a directed acyclic graph of the states, functions and
  messages of agents in the model
  \item stategraph\_colour.dot -- as above but functions are coloured
  \item process\_order\_graph.dot -- as above but the message syncronisation is
  shown
  \item latex.dot -- a latex document describing the model
\end{itemize}

The simulation program source code files:

\begin{itemize}
  \item Makefile -- the compilation script used by the program `make'
  \item xml.c -- the source code file that handles inputs and outputs of the
  simulation
  \item main.c -- the source code file containing the main program loop
  \item header.h -- a C header file for global variables and function
  declarations between source code files
  \item memory.c -- the source code file that handles the memory requirements
  of the simulation
  \item low\_primes.h -- holds data used for partitioning agents
  \item messageboards.c -- the source code that handles message functionality
  \item partitioning.c -- the source code that handles the partitioning of
  agents between nodes in parallel
  \item timing.c -- the source code that provides timing routines
  \item Doxyfile -- a configuration file for generating documentation using
 the program `doxygen'
  \item rules.c -- the source code file containing the generated rules for
  function conditions and message input filters
\end{itemize}

For each agent type an associated header file is created:

\begin{itemize}
\item <agent\_name>\_agent\_header.h -- the header file containing macros for
  accessing agent memory variables
\end{itemize}

The simulation source code files then require compilation, which can be easily
achieved using the included compilation script `Makefile' using the `make'
build automation tool. The program `make' invokes the `gcc' C compiler, which
are both free and available on various operating systems. If the parallel
version of the simulation was specified the compiler invoked by `make' is
`mpicc' which is a script usually available on parallel systems. To manually
change the C compiler used the parameter `CC' can be set:

\begin{verbatim}
  make CC=gcc
\end{verbatim}

To compile a model the message board library is required and the installation
directory is needed by the compilation script. Currently the directory for
`libmboard' is set in the same location of the model file. This can also be
manually changed by using the parameter `LIBMBOARD\_DIR':

\begin{verbatim}
  make LIBMBOARD_DIR=/location/of/libmboard
\end{verbatim}

The compiled program is called `main'. The parameters required to run a
simulation include the number of iterations to run for and the initial start
states (memory) of the agents, currently a formatted XML file.

\begin{verbatim}
  FLAME Application: test_model_simple_everthing 
  Debug mode enabled 
  Usage: ./main <number of iterations> [<states_directory>]/<init_state>
<partitions> [-f # | -f #+#]
  -f	Output frequency, 1st # is frequency, 2nd # is the offset if required
\end{verbatim}

The frequency of producing output files can also be set by the parameter `-f'.

\subsection{Start States Files}

The format of the initial start states XML is given by the following example:

%\begin{mylisting}
\begin{verbatim}
  <states>
  <itno>0</itno>
  
  <environment>
  <my_constant>6</my_constant>
  </environment>
  
  <agents>
    <xagent>
      <name>agent_name</name>
      <var_name>0</var_name>
      ...
    </xagent>
    ...
  </agents>

  </states>
\end{verbatim}
%\end{mylisting}

The root tag is called `states' and the `itno' tag holds the iteration number
that these states refer to. If there are any environment constants these are
placed within the `environment' tags. Any agents that exist are defined within
`xagent' tags and require the name of the agent within `name' tags. Any agent
memory variable (or environment constant) value is defined within tags with
the name of the variable. Arrays and data types are defined within curly
brackets with commas between each element.

When a simulation is running after every iteration, a states file is produced
in the same directory and in the same format as the start states file with the
values of each agent's memory.

For advanced usage one can use the `import' and `output' tags.

Import tags are used to include other locations of start state data. You have 
to provide the location (from where the current file is located), the format 
(currently only `xml' is avaliable) and the type (either `environment' data 
or `agent' data). Currently the import tags work to only one depth, they 
cannot be nested. The following import tags read in agent and environment data
from the file at `0\_depth\_1.xml' and agent data from
`../test7/0\_depth\_1\_agent.xml':

\clearpage

\begin{verbatim}
  <imports>
    <import>
      <location>0_depth_1.xml</location>
      <format>xml</format>
      <type>agent</type>
    </import>
    <import>
      <location>../test7/0_depth_1_agent.xml</location>
      <format>xml</format>
      <type>agent</type>
    </import>
    <import>
      <location>0_depth_1.xml</location>
      <format>xml</format>
      <type>environment</type>
    </import>
  </imports>
\end{verbatim}

Output tags are used to better define what output one wants to save from a 
simulation. Outputs can have two types, either `snapshot' (where the whole 
simulation state, environment and agents, is output) or `agent' where the 
agent name is specified and only that agent type is output. The format of the 
output can be specified (although only xml is provided currently) and the 
output location. Finally the timing can be specified. The `period' defines 
the number of iterations between producing output, for example a period of 10 
means data is output every 10 iterations, so if we start with iteration number 
0 the data will be output in iterations 10, 20, 30\ldots. A phase can also be
applied to the output timing. A phase of 2 when used with a period of 10 means
data is output 2 iterations after it usually would, for example it would now be
2, 12, 22, 32\ldots. The following example outputs a snapshot every 10
iterations and the agent `agent\_type\_a' every 10 iterations starting at the
second iteration:

\clearpage

\begin{verbatim}
  <outputs>
    <output>
      <type>snapshot</type>
      <format>xml</format>
      <location></location>
      <time>
        <period>10</period>
        <phase>0</phase>	
        </time>
    </output>
    <output>
      <type>agent</type>
      <name>agent_type_a</name>
      <format>xml</format>
      <location></location>
      <time>
        <period>10</period>
        <phase>2</phase>	
      </time>
    </output>
  </outputs>
\end{verbatim}

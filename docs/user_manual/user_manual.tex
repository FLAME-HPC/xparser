\documentclass[12pt,a4paper]{report}
\usepackage{t1enc}
\usepackage[latin1]{inputenc}
\usepackage[english]{babel}
\usepackage{url,graphics,lscape}
\usepackage{a4wide}
\usepackage{graphicx}
\usepackage{color}
%\usepackage{fancyheadings}
\usepackage{verbatim}

\newenvironment{mylisting}
{\begin{list}{}{\setlength{\leftmargin}{1em}}\item\small\bfseries}
{\end{list}}

\parskip 6pt         % sets spacing between paragraphs
\parindent 0pt       % sets leading space for paragraphs
%\pagestyle{fancy}
\begin{document}



\title{FLAME
\\User Manual}
\author{Simon Coakley\\Mariam Kiran
\\
\\ Unit - USFD}
%\date{17 May 2006}

\maketitle



\begin{abstract}
The report presents a manual for the FLAME framework. How to design a model,
create a description of the model, and write the implementation of the model.
Details are also included about how to execute a model.
\end{abstract}

\tableofcontents
\pagebreak

\section{Introduction}

The FLAME framework is an enabling tool to create agent-based models that can
be run on high performance computers (HPCs). Models are created based upon
extended finite state machines that include message inputs and outputs. This information
is used by the framework to automatically generate a simulation program that
can run models efficiently on HPCs.

\section{Model Design}
\label{sec:model_design}

The philosophy of FLAME is to specify an agent-based model as you would specify
software behaviour, as ultimately the execution of the model will be in software.
The behaviour model is based upon state machines which are composed of a number
of states with transition functions between those states. There is a single
start state and by traversing states using the transition functions the machine
executes the functions until it reaches an end state. This happens to each
machine as one time step or iteration is completed. Figure
\ref{fig:iteration_1} shows a model consisting of two agents each with two
functions run one after the other. A time step or iteration of the model is when
each agent goes from their start state to an end state.

\begin{figure}[ht]
\begin{center}
\includegraphics*[scale=0.5]{iteration_1.eps}
\caption{An iteration with 2 agents with 2 functions each}
\label{fig:iteration_1}
\end{center}
\end{figure}

Each agent has a memory that holds variables. Transition functions can read and
write to variables in the agent's memory. Communication between agents is
achieved via messages. Transition functions can also read incoming messages and write
outgoing messages. Figure \ref{fig:xmachine} shows an agent machine with a start state, 
an end state and one transition function from one to the other which has access to the 
agent memory and recieves inputs
and produces outputs.

\begin{figure}[ht]
\begin{center}
\includegraphics*[scale=0.5]{xmachine.eps}
\caption{An agent as a computational machine}
\label{fig:xmachine}
\end{center}
\end{figure}

\clearpage

Describing an agent-based model would thus include the following individual
stages for creating a model:

\begin{itemize}
\item Identifying the agents and their functions
\item Identify the states which impose some order of function execution
\item Identify the memory as the set of variables that are accessed by
functions \\(including possible conditions on variables for the functions to
occur)
\item Identify the input messages and output messages of each function 
\\(including possible filters on inputs)

\end{itemize}

Once a model has been defined using these criteria the implementation of the
agent functions can be written as source code in the C programming language.
FLAME can then use the model description to create a simulation program that
handles agent execution and communication in parallel.

\subsection{Swarm Example}

For example a simple swarm (flocking) model would include an agent for a bird.
Because agents can only communicate via messages in FLAME each bird needs to
have a function that sends out a message with their current location. A second
function is needed to read the messages and update the birds velocity depending
on the other birds locations. A third function then updates the birds location
using the birds new velocity. The three functions required of the bird agent are
then:

\begin{itemize}
\item signal -- send out current position message
\item observe -- read position messages from other agents and update velocity
\item respond -- update position using the current velocity
\end{itemize}

The functions occur in this order so states are included to impose this
order, see Figure \ref{fig:swarm_1}. As a requirement for automatic parallel
execution agents can only enter particular states once during an iteration,
i.e. there cannot be any loops back to a state already entered. This is so that
parallel processes can easily stay synchronised, adding to the efficiently of a
simulation. There can only be one start state per agent, but there can be many
possible end states.

\begin{figure}[ht]
\begin{center}
\includegraphics*[scale=0.5]{swarm_1.eps}
\caption{Swarm model including states}
\label{fig:swarm_1}
\end{center}
\end{figure}

Functions can also have conditions. For instance, in
the swarm model, a response function for flying and a response function when
resting on the ground. The condition on the flying response function would be that the
z-axis position of the agent be greater than zero while the resting response
function condition would be when the z-axis position was zero, see Figure
\ref{fig:swarm_2}.

\begin{figure}[ht]
\begin{center}
\includegraphics*[scale=0.5]{swarm_2.eps}
\caption{Swarm model including function conditions}
\label{fig:swarm_2}
\end{center}
\end{figure}

The messages required for communication between agents are a signal message,
which is output from `signal' and input to `observe', see Figure
\ref{fig:swarm_3}. This message would include the position of the agent that
sent it, see Table \ref{tab:signal_message}. A feature of swarm models and most
agent-based models is that there is generally a limit on incoming communication. 
In the swarm case this is the perceived distance of sight that an agent can view the location of other
agents. This feature can be added to the model as a filter on inputs to a
function, where the filter is a formula involving the position contained in the
message (the position of the sending agent) and the receiving agent position.

\begin{table}[ht]
\centering
\begin{tabular}{|l||c||l|}
\hline
Type&Name&Description\\
\hline \hline
double&px&x-axis position\\
\hline
double&py&y-axis position\\
\hline
double&pz&z-axis position\\
\hline
\end{tabular}
\caption{Signal Message}
\label{tab:signal_message}
\end{table}

\begin{figure}[ht]
\begin{center}
\includegraphics*[scale=0.5]{swarm_3.eps}
\caption{Swarm model including messages}
\label{fig:swarm_3}
\end{center}
\end{figure}

Functions that take a message type as input are only executed once all functions
that output the same message type have finished. One iteration is taken as a
standalone run of a simulation, so once all the functions that have a message
type as an input have been executed, the messages are deleted as they are no
longer required. Messages cannot be sent between iterations.

Finally the memory required by the agent functions include the position of the
agent, and its velocity, as shown in Table \ref{tab:swarm_memory}.

\begin{table}[ht]
\centering
\begin{tabular}{|l||c||l|}
\hline
Type&Name&Description\\
\hline \hline
double&px&position in x-axis\\
\hline
double&py&position in y-axis\\
\hline
double&pz&position in z-axis\\
\hline
double&vx&velocity in x-axis\\
\hline
double&vy&velocity in y-axis\\
\hline
double&vz&velocity in z-axis\\
\hline
\end{tabular}
\caption{Swarm Agent Memory}
\label{tab:swarm_memory}
\end{table}

The swarm model can also be represented as a transition table, see Table
\ref{tab:swarmtransition}, where:

\begin{itemize}
  \item Current State -- is the state the agent is currently in.
  \item Input -- is any inputs into the transition function.
  \item $M_{pre}$ -- are any preconditions of the memory on the transition.
  \item Function -- is the function name.
  \item $M_{post}$ -- is any change in the agent memory.
  \item Output -- is any outputs from the transition.
  \item Next State -- is the next state that is entered by the agent.
\end{itemize}

%\begin{landscape}
\begin{table}[ht]
\centering
\begin{tabular}{|c|c|c||c||c|c|c|}
\hline
Current State&Input&$M_{pre}$&Function&$M_{post}$&Output&Next State\\
\hline
\hline
start&&&signal&&signal&1\\
\hline
1&signal&&observe&(velocity updated)&&2\\
\hline
2&&$x > 0$&flying&(position updated)&&end\\
\hline
2&&$x == 0$&resting&(position updated)&&end\\
\hline
\end{tabular}
\caption{Swarm Agent Transition Table}
\label{tab:swarmtransition}
\end{table}
%\end{landscape}

Section \ref{sec:model_description} on model description describes how to write
a model description into an XML file that FLAME can understand. Section
\ref{sec:model_implementation} on model implementation describes how to
implement the individual agent functions, i.e. $M_{post}$ from the transition table.
Section \ref{sec:model_execution} on model execution describes how to use the
tools in FLAME to generate a simulation program, compile it, and run it.

% \begin{equation}\label{streamxmachine}
%     X = (\Sigma, \Gamma, Q, M, \Phi, F, q_{0}, m_{0})
% \end{equation}
% where,
% \begin{itemize}
% \item $\Sigma$ are the set of input alphabets
% \item $\Gamma$ are the set of output alphabets
% \item $Q$ denotes the set of states
% \item $M$ denotes the variables in the memory.
% \item $\Phi$ denotes the set of partial functions $\phi$ that map
% and input and memory variable to an output and a change on the
% memory variable. The set $\phi$: $\Sigma \times M\ \longrightarrow\
% \Gamma\times M$
% \item $F$ in the next state transition function, $F : Q \times\phi\longrightarrow
% Q$
% \item $q_{0}$ is the initial state and $m_{0}$ is the initial memory
% of the machine.
% \end{itemize}
% 
% \subsection{Transition Function}
% The transition functions allow the agents to change the state in
% which they are in, modifying their behaviour accordingly. These would
% require as inputs their current state $s_{1}$, current memory value
% $m_{1}$, and the possible arrival of a message that the agent is able to
% read, $t_{1}$. Depending on these three values the agent can then
% change to another state $s_{2}$, updates the memory to $m_{2}$ and
% optionally sends a message, $t_{2}$. Figure
% \ref{fig:trans} depicts how the transition function
% works within the agent.
% 
% % \begin{figure}
% % \begin{center}
% % \includegraphics*[scale=0.5]{transfn.eps}
% % \caption{Transition function} \label{fig:trans}
% % \end{center}
% % \end{figure}
% 
% 
% Extended finite state machines or X-Machines are used to define agents within a
% model. 
% The basic definition of an
% agent would thus, in accordance to the computational model, contain
% the following components:
% \begin{enumerate}
%  \item A finite set of internal states.
%  \item A set of transition functions that operate between states.
%  \item An internal memory set. In practice, the memory would be a finite set and can be structured in any way required.
%  \item A language for sending and receiving messages between other agents.
% \end{enumerate}
% 
% 
% Some of the transition functions may not depend on the incoming
% message. Thus the message would then be represented as:
% \begin{equation}\label{msg}
%     Message = \{ \emptyset, <data> \}
% \end{equation}
% 
% These agent transition functions may be expressed in terms of
% stochastic rules, thus allowing the multi-agent systems to be termed
% as stochastic systems.
% 
% \subsubsection{Memory and States}
% The difference between the internal set of states and the internal
% memory set allows for added flexibility when modelling systems.
% There can be agents with one internal state and all the complexity
% defined in the memory or equivalently, there could be agents with
% a trivial memory with the complexity then bound up in a large state
% space. There are good examples of choosing an appropriate balance
% between these two as this enables the complexity of the models to be
% better managed.

% \begin{figure}
% \begin{center}
% \includegraphics*[width = 4in]{X-Machine_agent.eps}
% \caption{X-Machine agent} \label{fig:xmachine}
% \end{center}
% \end{figure}


\section{Model Description}

Models descriptions are formatted in XML tag structures to allow
easy human and computer readability.

The DTD (Document Type Definition) of the XML document is currently located
here:

http://eurace.cs.bilgi.edu.tr/XMML.dtd

The start and end of a model file should be formatted as follows:

\begin{mylisting}
\begin{verbatim}
 <?xml version="1.0" encoding="ISO-8859-1"?>
 <!DOCTYPE xmodel SYSTEM "http://eurace.cs.bilgi.edu.tr/XMML.dtd">
 <xmodel version="2">
 <name>Model_name</name>
 <version>the version</version>
 <description>a description</description>
 ...
 </xmodel>
\end{verbatim}
\end{mylisting}

Models can contain:
\begin{itemize}
\item other models (enabled or disabled)
\item \textbf{environment}
\begin{itemize}
\item constant variables
\item function files
\item time units
% \begin{itemize}
% \item name
% \item *** unit
% \item *** period
% \end{itemize}
\item data types
% \begin{itemize}
% \item name
% \item description
% \item variables
% \end{itemize}
\end{itemize}
\item \textbf{agents}
\begin{itemize}
\item name
\item description
\item memory
% *** variables
\item functions
% *** name
% *** description
% *** current state
% *** next state
% *** condition
% *** inputs
% **** filter
% *** outputs
\end{itemize}
\item \textbf{messages}
\begin{itemize}
\item name
\item description
\item variables
\end{itemize}
\end{itemize}

\subsection{Environment}
The environment tag in the model.xml file hosts additional tags for information that may be required by
the parser for efficient simulation of the model.
Following are the tags that can be defined in it.


\subsubsection{Constant Variables}

Constant Variables refers to the global values used in the model.
These can me defined in a separate header file which can then be included in one of the functions file.

The header file would look as follows:

\begin{mylisting}
\begin{verbatim}
#define <varname> <value>
\end{verbatim}
\end{mylisting}

If this file was saved as a `my\_header.h' file, include this file into one of
the function files so that the compiler knows about these arguments.

\subsubsection{Function Files}

Function files are where you can place source code for the implementation of the agent functions.

They are included in the compilation script (Makefile) of the produced model.

\begin{mylisting}
\begin{verbatim}
 <functionFiles>
 <file>function_source_code_1.c</file>
 <file>function_source_code_2.c</file>
 </functionFiles>
\end{verbatim}
\end{mylisting}

\subsubsection{Time Rules}\label{timeunit}

Time rules allow the possibility of restricting the functions to
only execute during particular iterations. An iteration refers to
the smallest unit the models are set up on. In EURACE, we are
assuming every iteration to represent one day in the calender.

Time rules can be applied to function conditions instead of a
condition rule and are defined by a time period and a phase. A time
phase is the offset from the start of a period.




A time unit would contain:
\begin{itemize}
\item name - name of the time unit.
\item unit - can contain the iteration or other time units.
\item period - offset by which to skip the unit in every iteration.
\end{itemize}

A time period needs to be defined as a time unit in the environment
of a model. Time units can be described as:

\begin{mylisting}
\begin{verbatim}
<timeUnits>
  <timeUnit>
    <name>daily</name>
    <unit>iteration</unit>
    <period>1</period>
  </timeUnit>

  <timeUnit>
    <name>weekly</name>
    <unit>daily</unit>
    <period>5</period>
  </timeUnit>

  <timeUnit>
    <name>monthly</name>
    <unit>weekly</unit>
    <period>4</period>
  </timeUnit>

  <timeUnit>
    <name>quarterly</name>
    <unit>monthly</unit>
    <period>3</period>
  </timeUnit>

  <timeUnit>
    <name>yearly</name>
    <unit>monthly</unit>
    <period>12</period>
  </timeUnit>

</timeUnits>
\end{verbatim}
\end{mylisting}

A condition can then be added to the function definition to make the
parser aware that this particular function only has to be called at
certain times of the simulation. The condition can be added as
follows:
\begin{mylisting}
\begin{verbatim}
 <condition>
     <time>
     <period>monthly</period>
     <phase>a.day_of_month_to_act</phase>
     </time>
 </condition>
\end{verbatim}
\end{mylisting}

The condition allows the function to run \emph{monthly} at the phase
of \emph{day\_of\_month\_to\_act}. The
\emph{day\_of\_month\_to\_act} is a variable extracted from the
agent memory and is thus defined as
\emph{a.day\_of\_month\_to\_act}.

Refer to section \ref{functioncond} for more details on function
condition definitions.

These rules are then parsed into rule functions and placed in a file
called rules.c

\subsubsection{Data Types}

Data types are user defined data types that can be used in a model.

Data types can contain C data types or other predefined user data types.

\begin{mylisting}
\begin{verbatim}
<dataTypes>

 <dataType>
  <name>Histogram</name>
  <description>ADT Histogram</description>
  <variables>
   <variable><type>double</type><name>prob[30]</name><description></description>
   </variable>
   <variable><type>double</type><name>values[30]</name><description></description>
   </variable>
   <variable><type>double</type><name>max</name><description></description>
   </variable>
  </variables>
 </dataType>

 <dataType>
  <name>Belief</name>
  <description>ADT Belief</description>
  <variables>
   <variable><type>double</type><name>expectedPriceReturns</name><description></description>
   </variable>
   <variable><type>double</type><name>expectedTotalReturns</name><description></description>
   </variable>
   <variable><type>double</type><name>expectedCashFlowYield</name><description></description>
   </variable>
   <variable><type>double</type><name>volatility</name><description></description>
   </variable>
   <variable><type>Histogram</type><name>hist</name><description></description>
   </variable>
  </variables>
 </dataType>

</dataTypes>
\end{verbatim}
\end{mylisting}

In the example above the datatype \emph{Belief} contains a variable
of datatype \emph{Histogram}

\subsection{Agent}

An agent can be defined as follows:
\begin{mylisting}
\begin{verbatim}
<agents>

  <xagent>
    <name>Firm</name>
    <description></description>

    <memory>
      <variable><type>int</type><name>id</name><description></description></variable>
      <variable><type>int</type><name>region_id</name><description></description></variable>
      <variable><type>int</type><name>gov_id</name><description></description></variable>
      <variable><type>int</type><name>day_of_month_to_act</name><description></description></variable>
      <variable><type>double</type><name>payment_account</name><description></description></variable>
    </memory>

    <functions>
      .....
    </functions>

  </xagent>

  <xagent>
    <name>Household</name>
    <description></description>

    <memory>
      <variable><type>int</type><name>id</name><description></description></variable>
      <variable><type>int</type><name>region_id</name><description></description></variable>
      <variable><type>int_array</type><name>neighboring_region_ids</name><description></description></variable>
      <variable><type>int</type><name>gov_id</name><description></description></variable>
      <variable><type>int</type><name>day_of_month_to_act</name><description></description></variable>
      <variable><type>double</type><name>payment_account</name><description></description></variable>
    </memory>

    <functions>
      <function>
        <name>Household_read_firing_messages</name>
        <description>The household checks whether is is fired or not</description>
        <currentState>EXIT_FINANCIAL_MARKET</currentState>
        <nextState>01d</nextState>
        <condition><lhs><value>a.employee_firm_id</value></lhs><op>NEQ</op><rhs><value>-1</value></rhs></condition>
        <inputs>
          <input><messageName>firing</messageName></input>
        </inputs>
      </function>
    </functions>

  </xagent>
</agents>
\end{verbatim}
\end{mylisting}

\subsubsection{Agent Functions}
An agent function contains:
\begin{itemize}
\item name
\item description
\item current state - The current state point the agent is in, in
the branch.
\item next state - the next state point it has to move to, in the
branch.
\item condition - condition which tells the parser when this
function can be executed.
\item inputs - the messages this function is reading.
\item outputs - the messages this function is sending.
\end{itemize}

\subsubsection{Function Condition and Message Input Filter Rule
Tags}\label{functioncond}

The agent functions can be accompanied by conditions telling the
parser when these functions have to be executed during the
iterations. Condition on the basis of time have been explained in
section \ref{timeunit} but these can also be associated with
comparison rules.

\paragraph{Comparison Rules}

\begin{mylisting}
\begin{verbatim}
<lhs></lhs><op></op><rhs></rhs>
\end{verbatim}
\end{mylisting}

lhs and rhs can be either a value, denoted by value tags:

\begin{mylisting}
\begin{verbatim}
<value></value>
\end{verbatim}
\end{mylisting}

or another rule.

Values can include agent and message memory variables, which are denoted by either:

\begin{mylisting}
\begin{verbatim}
a.agent_var
m.message_var
\end{verbatim}
\end{mylisting}

op can be either comparison functions:

\begin{itemize}
\item EQ -- equal to
\item NEQ -- not equal to
\item LEQ -- less than or equal to
\item GEQ -- greater than or equal to
\item LT -- less then
\item GT -- greater than
\end{itemize}
or logic operators:
\begin{itemize}
\item AND
\item OR
\end{itemize}
the operator NOT is used by placing `not' tags around a rule:
\begin{mylisting}
\begin{verbatim}
<condition>
 <lhs>
  <lhs><value>a.employee_firm_id</value></lhs>
  <op>GT</op>
  <rhs><value>-1</value></rhs>
 </lhs>
 <op>AND</op>
 <rhs>
  <not>
  <lhs><value>a.on_the_job_search</value></lhs>
  <op>EQ</op>
  <rhs><value>0</value></rhs>
  </not>
 </rhs>
</condition>
\end{verbatim}
\end{mylisting}

\paragraph{Message Filter}
Message filters allow the messages to be filtered before being
provided to the function. This allows the messages to be checked
according to a condition before being read by the function. The
message filter can be added as follows:

\begin{mylisting}
\begin{verbatim}
<input>
 <messageName>firing</messageName>
 <filter>
  <lhs><value>a.id</value></lhs>
  <op>EQ</op>
  <rhs><value>m.worker_id</value></rhs>
 </filter>
</input>
\end{verbatim}
\end{mylisting}

Thus in the above example messages will be filtered according to the
message variable \emph{worker\_id} (defined as m.<varname>) to be EQ
(equal) to the agent \emph{id} (defined as a.<varname>).

\subsection{Messages}
The agents are communicating with each other through messages being
sent in the system. The messages being processed have to be defined
in the XML file. All message lists are emptied at the end of the
iteration.

\begin{mylisting}
\begin{verbatim}
<messages>

 <message>
  <name>bank_account_update</name>
  <description>Sent by household. Household informs the bank about the actual payment account</description>
  <variables>
    <variable><type>int</type><name>id</name><description></description></variable>
    <variable><type>int</type><name>bank_id</name><description></description></variable>
    <variable><type>double</type><name>payment_account</name><description></description></variable>
  </variables>
 </message>

 <message>
  <name>application_rejection</name>
  <description>Send by firms. Includes the id and the id of the refused applicant.</description>
  <variables>
    <variable><type>int</type><name>firm_id</name><description></description></variable>
    <variable><type>int</type><name>worker_id</name><description></description></variable>
  </variables>
 </message>

</messages>
\end{verbatim}
\end{mylisting}

\section{Model Implementation}
\label{model_implementation}

The implementations of each agent's functions are currently written in separate
files written in C, usually suffixed with `.c'. Each file must include two
header files, one for the overall framework and one for the particular agent that the functions are for.
Functions for different agents cannot be contained in the same file.
Thus, at the top of each file two headers are required:

\begin{mylisting}
\begin{verbatim}
#include "header.h"
#include "<agentname>_agent_header.h"
\end{verbatim}
\end{mylisting}

Where `<agent\_name>' is replaced with the actual agent name.
Agent functions can then be written in the following style:

\begin{mylisting}
\begin{verbatim}
/*
 * \fn: int function_name()
 * \brief: A brief description of the function.
 */
int function_name()
{
   /* Function code here */

   return 0; /* Returning zero means the agent is not removed */
}
\end{verbatim}
\end{mylisting}

The first commented part (four lines) is good practice and can be used to
auto-generate source code documentation. The function name should coordinate
with the agent function name and the function should return an integer. The
functions have no parameters. Returning zero means the agent is not removed from
the simulation, and one removes the agent immediately from the simulation.

\subsection{Accessing Agent Memory Variables}

After including the specific agent header, the variables in the
agent memory can be accessed by capitalising the variable name.

\begin{mylisting}
\begin{verbatim}
AGENT_VARIABLE
\end{verbatim}
\end{mylisting}

To access elements of a static array just use square brackets and index number as normal.

\begin{mylisting}
\begin{verbatim}
MY_STATIC_ARRAY[index]
\end{verbatim}
\end{mylisting}

To access the elements and the size of dynamic array variables use
`.size' and `.array[index]'

\begin{mylisting}
\begin{verbatim}
MY_DYNAMIC_ARRAY.size
MY_DYNAMIC_ARRAY.array[i]
\end{verbatim}
\end{mylisting}

To access variables of a model data type use `.variablename'

\begin{mylisting}
\begin{verbatim}
MY_DATA_TYPE.variablename
\end{verbatim}
\end{mylisting}

\subsubsection{Using Model Data Types}

Following is an example of how to use a data type called
\emph{vacancy}.

\begin{mylisting}
\begin{verbatim}
/* Allocate own data type */
vacancy vac;
/* And initialise */
init_vacancy(&vac);
/* Initialise a static array of the data type */
init_vacancy_static_array(&vac_static_array, array_size);
/* Free a data type */
free_vacancy(&vac);
/* Free a static array of a data type */
free_vacancy_static_array(&vac_static_array, array_size);
/* Copy a data type */
copy_vacancy(&vac_from, &vac_to);
/* Copy a static array of a data type */
copy_vacancy_static_array(&vac_static_array_from,
                          &vac_static_array_to, array_size);
\end{verbatim}
\end{mylisting}

If the data type is a variable from the agent memory, then the data type
variable name must be capitalised.

\subsubsection{Using Dynamic Arrays}

Dynamic array types are created by adding `\_array' to a data type.
When passing a dynamic array variable to the following functions
place an \& in front of the array.

\begin{mylisting}
\begin{verbatim}
/* Allocate own array */
vacancy_array vacancy_list;

/* And initialise */
init_vacancy_array(&vacancy_list);

/* Reset an array */
reset_vacancy_array(&vacancy_list);

/* Free an array */
free_vacancy_array(&vacancy_list);

/* Add an element to the array */
add_vacancy(&vacancy_list, var1,
var2, var3);

/* Remove an element at index index */
remove_vacancy(&vacancy_list,
index);

/* Copy the array */

copy_vacancy_array(&from_list, &to_list);
\end{verbatim}
\end{mylisting}

If the dynamic array is a variable from the agent memory, then the dynamic
array variable name must be capitalised.

\subsection{Sending and receiving messages}

Messages can be read using macros to loop through the incoming message list as per the template below.
Message variables can be accessed using an arrow `->'

\begin{mylisting}
\begin{verbatim}
START_MESSAGENAME_MESSAGE_LOOP
 messagename_message->variablename
FINISH_MESSAGENAME_MESSAGE_LOOP
\end{verbatim}
\end{mylisting}

Messages are sent or added to the message list by
\begin{mylisting}
\begin{verbatim}
 add_messagename_message(var1, var2);
\end{verbatim}
\end{mylisting}

\section{Model Execution}
\label{model_execution}

FLAME contains a parser program called `xparser' that parses a model
XML definition into simulation program source code that can be compiled
together with model implementation source code. The xparser includes
template files which are used to generate the simulation program source code.

The xparser takes as parameters the location of the model file and an option
for serial or parallel (MPI) version, serial being the default if the option is
not specified.

\subsection{Generated Files}

The xparser then generates simulation source code files in the same directory
as the model file. These files include:

\begin{itemize}
  \item Doxyfile -- a configuration file for generating documentation using
 the program `doxygen'
  \item header.h -- a C header file for global variables and function
  declarations between source code files
  \item low\_primes.h -- holds data used for partitioning agents
  \item main.c -- the source code file containing the main program loop
  \item Makefile -- the compilation script used by the program `make'
  \item memory.c -- the source code file that handles the memory requirements
  of the simulation
  \item xml.c -- the source code file that handles inputs and outputs of the
  simulation
  \item <agent\_name>\_agent\_header.h -- the header file containing macros for
  accessing agent memory variables
  \item rules.c -- the source code file containing the generated rules for
  function conditions and message input filters
  \item messageboards.c -- deprecated?
  \item partitioning.c -- still used?
\end{itemize}

and in parallel the additional files:

\begin{itemize}
  \item propagate\_messages.c -- deprecated?
  \item propagate\_agents.c -- still used?
\end{itemize}

The simulation source code files then require compilation, which can be easily
achieved using the included compilation script `Makefile' using the `make'
build automation tool. The program `make' invokes the `gcc' C compiler, which
are both free and available on various operating systems. If the parallel
version of the simulation was specified the compiler invoked by `make' is
`mpicc' which is a script usually available on parallel systems.

The compiled program is called `main'. The parameters required to run a
simulation include the number of iterations to run for and the initial start
states (memory) of the agents, currently a formatted XML file.

\subsection{Start States Files}

The format of the initial start states XML is given by the following example:

\begin{mylisting}
\begin{verbatim}
<states>
<itno>0</itno>

<environment>
<my_constant>6</my_constant>
</environment>

<xagent>
<name>agent_name</name>
<var_name>0</var_name>
...
</xagent>

...

</states>
\end{verbatim}
\end{mylisting}

The root tag is called `states' and the `itno' tag holds the iteration number
that these states refer to. If there are any environment constants these are
placed within the `environment' tags. Any agents that exist are defined within
`xagent' tags and require the name of the agent within `name' tags. Any agent
memory variable (or environment constant) value is defined within tags with
the name of the variable. Arrays and data types are defined within curly
brackets with commas between each element.

When a simulation is running after every iteration, a states file is produced
in the same directory and in the same format as the start states file with the
values of each agent's memory.


\appendix

\section{XML DTD}                       % B
\label{cha_xmldtd}

\small{{\tt \verbatiminput{xmml.dtd}}}


\bibliographystyle{alpha}
\bibliography{eurace_refs}

\end{document}
